\section{Question 2}
\subsection{Part 1}
Suppose in PCA we project each point \( x_i \) to \( z_i = V_k^T x_i \), where \( V_k = [v_1, \dots, v_k] \), i.e., the first \( k \) principal components. We can reconstruct \( x_i \) from \( z_i \) as follows:

\[
\hat{x_i} = V_k z_i
\]

Show that

\[
\| x_i - \hat{x_i} \| = \| z_i - z_j \|
\]

holds.
\begin{qsolve}
    \begin{qsolve}[]
        from the given information, we have:
        \[
        \hat{x}_i = V_{1:k} z_i
        \]
        \[
        \hat{x}_j = V_{1:k} z_j
        \]
        so we can write:
        \[
        \hat{x}_i - \hat{x}_j = V_{1:k} z_i - V_{1:k} z_j
        \]
        \[
        \hat{x}_i - \hat{x}_j = V_{1:k} (z_i - z_j)
        \]
        thus we can conclude that:
        \[
        \| \hat{x}_i - \hat{x}_j \|^2 = \| V_{1:k} (z_i - z_j) \|^2
        \]
        so if we expand the right side of the equation by the fact that $\| x \|^2 = x^T x$ we get:
        \[
        \| V_{1:k} (z_i - z_j) \|^2 = (V_{1:k} (z_i - z_j))^T (V_{1:k} (z_i - z_j))
        \]
        \[
        = (z_i - z_j)^T V_{1:k}^T V_{1:k} (z_i - z_j)
        \]
        Since $V_{1:k}$ consists of the first $k$ principal components, its columns are orthonormal.
        \[
        \Rightarrow V_{1:k}^T V_{1:k} = I
        \]
        \[
        \Rightarrow (z_i - z_j)^T V_{1:k}^T V_{1:k} (z_i - z_j) = (z_i - z_j)^T I (z_i - z_j)
        \]
        \[
        = (z_i - z_j)^T (z_i - z_j)
        \]
        \[
        = \| z_i - z_j \|^2
        \]
        thus we approve that:
        \begin{center}
            \hl{$\Rightarrow \| \hat{x}_i - \hat{x}_j \|^2 = \| z_i - z_j \|^2$}
        \end{center}
        
    \end{qsolve}
\end{qsolve}
\subsection{Part 2}
Show that the reconstruction error is equal to:

\[
\sum_{i=1}^{n} \| x_i - \hat{x_i} \|^2 = (n-1) \sum_{j=k+1}^{p} \lambda_j
\]

where \( \lambda_{k+1}, \dots, \lambda_p \) are the smallest eigenvalues. Therefore, the more principal components we use for reconstruction, the better the accuracy.
\begin{qsolve}
    \begin{qsolve}[]
        again from the given information, we have:
        \[
        \hat{x_i} = V_{1:k}V_{1:k}^Tx_i
        \]
        so
        \[
        x_i - \hat{x_i} = x_i - V_{1:k}V_{1:k}^Tx_i
        \]
        this can be written as:
        \[
        x_i - \hat{x_i} = (I - V_{1:k}V_{1:k}^T)x_i
        \]
        and if we get norm of the above equation we get:
        \[
        \|x_i - \hat{x_i}\|^2 = \|(I - V_{1:k}V_{1:k}^T)x_i\|^2
        \]
        Sum of Squared Reconstruction Errors can be written as:
        \[
        \sum_{i=1}^{n}\|x_i - \hat{x_i}\|^2 = \sum_{i=1}^{n}\|(I - V_{1:k}V_{1:k}^T)x_i\|^2
        \]
        $$ \sum_{i=1}^{n}\|(I - V_{1:k}V_{1:k}^T)x_i\|^2 = \sum_{i=1}^{n}\|V_{k+1:p}V_{k+1:p}^Tx_i\|^2 $$
        $$ = \sum_{i=1}^{N}(V_{k+1:p}V_{k+1:p}^T x_i)^T(V_{k+1:p}V_{k+1:p}^T x_i) $$
        $$ = \sum_{i=1}^{N}x_i^T V_{k+1:p}V_{k+1:p}^T V_{k+1:p}V_{k+1:p}^T x_i $$
        so cause $V_{k+1:p}$ is orthogonal matrix, we have:
        $$ = \sum_{i=1}^{N}x_i^T V_{k+1:p}V_{k+1:p}^T x_i $$
        $$ = \sum_{i=1}^{N}Tr(x_i^T V_{k+1:p}V_{k+1:p}^T x_i) $$
        \splitqsolve[\splitqsolve]
		$$ = \sum_{i=1}^{N}Tr(V_{k+1:p}^T x_i x_i^T V_{k+1:p}) $$
		so as we know we can define $S = \frac{1}{N-1}\sum_{i=1}^{N}(x_i-\bar{x})(x_i-\bar{x})^T$ and as $\bar{x} = 0$ this can be written as:
        $$ S = \frac{1}{N-1}\sum_{i=1}^{N}x_i x_i^T $$
        so we can write:
        $$ \sum_{i=1}^{N}Tr(V_{k+1:p}^T x_i x_i^T V_{k+1:p}) = Tr(V_{k+1:p}^T S V_{k+1:p}) $$
        which is equal to:
        $$ (n-1)Tr(V_{k+1:p}^T S V_{k+1:p}) = (n-1)\sum_{i=k+1}^{p}\lambda_i $$
        so the sum of squared reconstruction errors is equal to:
        \begin{center}
            \hl{$ \sum_{i=1}^n \| x_i - \hat{x}_i \|_2^2 = (n - 1) \sum_{i=k+1}^p \lambda_i$}
        \end{center}
        The equation
        $
        \sum_{i=1}^{n} \|x_i - \hat{x}_i\|^2 = (n-1) \sum_{i=k+1}^{p} \lambda_i
        $
        provides insights into how PCA minimizes reconstruction error by focusing on the largest eigenvalues (principal components) that capture the most variance in the data. It highlights the trade-offs involved in dimensionality reduction and reconstruction accuracy in PCA.The reconstruction error is directly related to
        the eigenvalues that correspond to the discarded principal components. The larger these
        eigenvalues, the greater the reconstruction error. If we use more principal components (larger k) to reconstruct the data, the reconstruction error will decrease.
    \end{qsolve}
\end{qsolve}