\section{Question 4}
\subsection{Part 1}
Consider a linear regression problem that includes \( n \) data points and \( d \) features. When \( n = d \), the matrix \( F \in \mathbb{R}^{n \times n} \) has the biggest eigenvalue \( \alpha \) and the smallest eigenvalue with a very small value. We have \( y = Fw^* + \epsilon \). If we calculate \( \hat{w}_{inv} = F^{-1} y \), cause of small singular value of F and having noise we see that \( \|\hat{w}_{inv} - w^*\| = 10^{10} \).

Instead of inverting \( F \), assume we use gradient descent. We repeat gradient descent \( k \) times starting from \( w = 0 \) with a loss function \( \ell(w) = \frac{1}{2} \| y - Fw \|^2 \). We assume that the learning rate \( \eta \) is small enough to ensure the stability of gradient descent for the given problem (this is an important point).

The gradient descent update formula for \( t > 0 \) is as follows:
\[
w_t = w_{t-1} - \eta \left( F^T \left( F w_{t-1} - y \right) \right)
\]

We are looking for the error \( \| w_k - w^* \|_2 \). We want to show that, in the worst case, this error can be bounded by the following:
\[
\| w_k - w^* \|_2 \leq k \eta \alpha \| y\|_2 + \| \hat{w}_l \|_2
\]
In other words, the error cannot go out of bounds, at least not too quickly.

To complete this task, we only need to prove the key idea using the triangle inequality and the norm properties, as the result will follow naturally.

Show that for \( t > 0 \):
\[
\| w_t \|_2 \leq \| w_{t-1} \|_2 + \eta \alpha \| y \|_2
\]

\begin{qsolve}
	\begin{qsolve}[]
		The gradient descent update rule is:
		$$
		w_t = w_{t-1} - \eta F^T (F w_{t-1} - y)
		$$
		This can be expanded as:
		$$
		w_t = w_{t-1} - \eta F^T F w_{t-1} + \eta F^T y
		$$
		if we factorize \( w_{t-1} \) from the equation, we can simplify the equation as:
		$$
		w_t = (I - \eta F^T F) w_{t-1} + \eta F^T y
		$$
		now we take the norm of both sides:
		$$
		\| w_t \|_2 = \| (I - \eta F^T F) w_{t-1} + \eta F^T y \|_2
		$$
		if we use the triangle inequality, we can bound the above equation as:
		\splitqsolve[\splitqsolve]
		$$
		\| w_t \|_2 \leq \left\| (I - \eta F^T F) w_{t-1} \right\|_2 + \left\| \eta F^T y \right\|_2
		$$
		now we have 2 terms in the right side of the equation, we can bound each term separately:
		$$
		\left\| (I - \eta F^T F) w_{t-1} \right\|_2 \leq \left\| I - \eta F^T F \right\|_2 \left\| w_{t-1} \right\|_2
		$$
		The spectral norm \( \left\| I - \eta F^T F \right\|_2 \) can be bounded using the eigenvalues of \( F^T F \). Let \( \lambda_{\min} \) and \( \lambda_{\max} \) be the smallest and largest eigenvalues of \( F^T F \), respectively.
		Since \( \eta \) is small enough to ensure stability, and the smallest eigenvalue is very small (\( \lambda_{\min} \approx 0 \)), we have:
		$$
		\left\| I - \eta F^T F \right\|_2 = \max_{i} \left|1 - \eta \lambda_i\right| = 1 - \eta \lambda_{\min} \approx 1
		$$
		so we can bound the first term as:
		$$
		\left\| (I - \eta F^T F) w_{t-1} \right\|_2 \leq \left\| w_{t-1} \right\|_2 \quad (1)
		$$
		Now we bound the second term:
		$$
		\left\| \eta F^T y \right\|_2 = \eta \left\| F^T y \right\|_2 \leq \eta \left\| F^T \right\|_2 \left\| y \right\|_2
		$$
		Since \( \left\| F^T \right\|_2 = \left\| F \right\|_2 \), and the largest eigenvalue of \( F \) is \( \alpha \):
		$$
		\left\| \eta F^T y \right\|_2 \leq \eta \alpha \left\| y \right\|_2 \quad (2)
		$$
		Combining equations (1) and (2), we get:
		$$
		\| w_t \|_2 \leq \| w_{t-1} \|_2 + \eta \alpha \| y \|_2
		$$
		and this completes the proof.
	\end{qsolve}
\end{qsolve}
If gradient descent cannot diverge, what can be said about the eigenvalues of \( (I - \eta F^T F) \), what shape do the eigenvalues take?
\begin{qsolve}
	\begin{qsolve}[]
		Let \( \lambda_i \) be the eigenvalues of \( F^T F \). Since \( F^T F \) is a symmetric positive semi-definite matrix, all its eigenvalues are non-negative:
		$$
		\lambda_i \geq 0 \quad \text{for all } i.
		$$
		The eigenvalues of the matrix \( I - \eta F^T F \) are given by:
		$$
		\mu_i = 1 - \eta \lambda_i.
		$$
		For the gradient descent algorithm to \textbf{not diverge}, the magnitude of each eigenvalue \( \mu_i \) must be less than or equal to 1:
		\splitqsolve[\splitqsolve]
		$$
		| \mu_i | \leq 1.
		$$
		This condition ensures that the iterative updates do not amplify the errors, keeping the algorithm stable. now if we substitute the expression for \( \mu_i \) in the above inequality, we get:
		$$
		|1 - \eta \lambda_i| \leq 1 \implies -1 \leq 1 - \eta \lambda_i \leq 1
		$$
		which simplifies to:
		$$
		0 \leq \eta \lambda_i \leq 2.
		$$
		so we can say that $0 \leq \eta \leq \frac{2}{\lambda_i}$, and this can simplify to:
		$$
		0 \leq \eta \leq \frac{2}{\lambda_{max}}
		$$
		If gradient descent cannot diverge, the eigenvalues of \( (I - \eta F^T F) \) must be real numbers within the interval \( [ -1, 1 ] \). They take the shape of values that, when applied iteratively, do not cause the weight vector to grow unboundedly. This ensures the stability and convergence of the gradient descent algorithm.
	\end{qsolve}
\end{qsolve}