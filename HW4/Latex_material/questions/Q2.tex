\section{Question 2}

Mention two issues that exist when using the one-hot vector for word representation.
\begin{qsolve}
    \begin{qsolve}[]
        Firstly, one-hot encoding results in very high-dimensional vectors. The size of the vector depends on the vocabulary size, which can be enormous for natural language datasets. This high dimensionality makes computations inefficient and requires a lot of memory, which is particularly problematic for tasks involving large vocabularies.

        Secondly, one-hot vectors do not capture any semantic or contextual relationships between words. For instance, the words "king" and "queen" are treated as entirely independent and unrelated, even though they share a semantic connection. This lack of meaning in the representation severely limits the model's ability to generalize and understand linguistic patterns.
        
        These limitations highlight the need for more efficient and meaningful word representations, such as dense embeddings learned through methods like Word2Vec or GloVe.
    \end{qsolve}
\end{qsolve}