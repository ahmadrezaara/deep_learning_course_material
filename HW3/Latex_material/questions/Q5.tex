\section{Question 5 (40 points)}
The error function in a network with Gaussian Dropout applied is described as follows:

\[
J_1 = 0.5 \left( y_d - \sum_{k=1}^n \delta_k W_k x_k \right)^2
\]

where \( \delta_k \sim \mathcal{N}(1, \sigma^2) \). Simplify the expected value of the gradient of the objective function with respect to the variable \( W_i \) as much as possible.
\begin{qsolve}
	\begin{qsolve}[]
		The gradient of \( J_1 \) with respect to \( W_i \) is:
		\[
		\frac{\partial J_1}{\partial W_i} = -\delta_i x_i \left( y_d - \sum_{k=1}^n \delta_k W_k x_k \right).
		\]

		Now, we compute \( \mathbb{E} \left[ \frac{\partial J_1}{\partial W_i} \right] \):
		\[
		\mathbb{E} \left[ \frac{\partial J_1}{\partial W_i} \right] = \mathbb{E} \left[ -\delta_i x_i \left( y_d - \sum_{k=1}^n \delta_k W_k x_k \right) \right].
		\]
		using the linearity of expectation, we can expand this to:
		\[
		\mathbb{E} \left[ \frac{\partial J_1}{\partial W_i} \right] = -x_i \left( \mathbb{E} \left[ \delta_i \right] y_d - \sum_{k=1}^n W_k x_k \mathbb{E} \left[ \delta_i \delta_k \right] \right).
		\]
		first we compute \( \mathbb{E} \left[ \delta_i \right] \) and \( \mathbb{E} \left[ \delta_i \delta_k \right] \). as it is mentioned in the question, \( \delta_i \sim \mathcal{N}(1, \sigma^2) \).so we have:
		\[
		\mathbb{E} \left[ \delta_i \right] = 1
		\]
		now for computing \( \mathbb{E} \left[ \delta_i \delta_k \right] \), we have two cases. if \( i = k \), then:
		\[
		\mathbb{E} \left[ \delta_i \delta_k \right] = \mathbb{E} \left[ \delta_i^2 \right] = \text{Var}(\delta_i) + \mathbb{E} \left[ \delta_i \right]^2 = \sigma^2 + 1.
		\]
		if \( i \neq k \), then:
		\[
		\mathbb{E} \left[ \delta_i \delta_k \right] = \mathbb{E} \left[ \delta_i \right] \mathbb{E} \left[ \delta_k \right] = 1 \times 1 = 1.
		\]
		the last term comes from the fact that the two random variables are independent.
		so we have:
		\[
		\mathbb{E} \left[ \delta_i \delta_k \right] = \begin{cases}
		\sigma^2 + 1 & \text{if } i = k, \\
		1 & \text{if } i \neq k.
		\end{cases}
		\]
		Substitute these values back into the expression:
		\splitqsolve[\splitqsolve]
		\[
		\mathbb{E} \left[ \frac{\partial J_1}{\partial W_i} \right] = -x_i \left( y_d - \sum_{k=1}^n W_k x_k \mathbb{E} \left[ \delta_i \delta_k \right] \right).
		\]
		Or equivalently:
		\[
		\mathbb{E} \left[ \frac{\partial J_1}{\partial W_i} \right] = -x_i \left( y_d - W_i x_i (\sigma^2 + 1) - \sum_{\substack{k=1 \\ k \neq i}}^n W_k x_k \right).
		\]
	\end{qsolve}
\end{qsolve}
Can you define a form of regularization using this type of Dropout? If so, introduce both Regularized and Non-Regularized targets accordingly.
\begin{qsolve}
	\begin{qsolve}[]
		Without regularization, the objective function is simply the squared error function \( J_1 \), which we are trying to minimize. It is given by:
		\[
		J_1 = 0.5 \left( y_d - \sum_{k=1}^n W_k x_k \right)^2,
		\]
		
		When Gaussian Dropout is applied, the objective function becomes as follows which $\delta_k \sim \mathcal{N}(1, \sigma^2)$: 
		\[
		J_{1,\text{reg}} = 0.5 \mathbb{E}_{\delta} \left[ \left( y_d - \sum_{k=1}^n \delta_k W_k x_k \right)^2 \right],
		\]

		The expectation in \( J_{1,\text{reg}} \) can be expanded as follows:

		\[
		J_{1,\text{reg}} = 0.5 \left( y_d - \sum_{k=1}^n W_k x_k \right)^2 + 0.5 \sigma^2 \sum_{k=1}^n W_k^2 x_k^2.
		\]

		
		The first term, \( 0.5 \left( y_d - \sum_{k=1}^n W_k x_k \right)^2 \), is the original (non-regularized) objective function.
		
		The second term, \( 0.5 \sigma^2 \sum_{k=1}^n W_k^2 x_k^2 \), represents the regularization term added by Gaussian Dropout.
		Non-Regularized Target can be shown as:
		\[
		J_1 = 0.5 \left( y_d - \sum_{k=1}^n W_k x_k \right)^2.
		\]

		and Regularized Target (with Gaussian Dropout) as:
		\[
		J_{1,\text{reg}} = 0.5 \left( y_d - \sum_{k=1}^n W_k x_k \right)^2 + 0.5 \sigma^2 \sum_{k=1}^n W_k^2 x_k^2.
		\]
		

	\end{qsolve}
\end{qsolve}